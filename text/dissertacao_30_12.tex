\documentclass[12 pt,]{article}
\usepackage[left=1in,top=1in,right=1in,bottom=1in]{geometry}
\newcommand*{\authorfont}{\fontfamily{phv}\selectfont}
\usepackage[]{mathpazo}


  \usepackage[T1]{fontenc}
  \usepackage[utf8]{inputenc}



\usepackage{abstract}
\renewcommand{\abstractname}{}    % clear the title
\renewcommand{\absnamepos}{empty} % originally center

\renewenvironment{abstract}
 {{%
    \setlength{\leftmargin}{0mm}
    \setlength{\rightmargin}{\leftmargin}%
  }%
  \relax}
 {\endlist}

\makeatletter
\def\@maketitle{%
  \newpage
%  \null
%  \vskip 2em%
%  \begin{center}%
  \let \footnote \thanks
    {\fontsize{18}{20}\selectfont\raggedright  \setlength{\parindent}{0pt} \@title \par}%
}
%\fi
\makeatother




\setcounter{secnumdepth}{0}



\title{Discurso Autoritário e Atitudes Políticas: um estudo do caso brasileiro.  }



\author{\Large Marcio de Lucas\vspace{0.05in} \newline\normalsize\emph{Universidade Federal de Pernambuco}  }


\date{}

\usepackage{titlesec}

\titleformat*{\section}{\normalsize\bfseries}
\titleformat*{\subsection}{\normalsize\itshape}
\titleformat*{\subsubsection}{\normalsize\itshape}
\titleformat*{\paragraph}{\normalsize\itshape}
\titleformat*{\subparagraph}{\normalsize\itshape}




\newtheorem{hypothesis}{Hypothesis}
\usepackage{setspace}

\makeatletter
\@ifpackageloaded{hyperref}{}{%
\ifxetex
  \PassOptionsToPackage{hyphens}{url}\usepackage[setpagesize=false, % page size defined by xetex
              unicode=false, % unicode breaks when used with xetex
              xetex]{hyperref}
\else
  \PassOptionsToPackage{hyphens}{url}\usepackage[unicode=true]{hyperref}
\fi
}

\@ifpackageloaded{color}{
    \PassOptionsToPackage{usenames,dvipsnames}{color}
}{%
    \usepackage[usenames,dvipsnames]{color}
}
\makeatother
\hypersetup{breaklinks=true,
            bookmarks=true,
            pdfauthor={Marcio de Lucas (Universidade Federal de Pernambuco)},
             pdfkeywords = {},  
            pdftitle={Discurso Autoritário e Atitudes Políticas: um estudo do caso brasileiro.},
            colorlinks=true,
            citecolor=blue,
            urlcolor=blue,
            linkcolor=magenta,
            pdfborder={0 0 0}}
\urlstyle{same}  % don't use monospace font for urls

% set default figure placement to htbp
\makeatletter
\def\fps@figure{htbp}
\makeatother

\usepackage{indentfirst}
\usepackage{setspace}
\setlength\parindent{1.3cm}
\setlength\parskip{0.3cm}
\usepackage{amsmath}
\usepackage{graphicx}
\usepackage{natbib}
\bibliographystyle{abnt}


% add tightlist ----------
\providecommand{\tightlist}{%
\setlength{\itemsep}{0pt}\setlength{\parskip}{0pt}}

\begin{document}
	
% \pagenumbering{arabic}% resets `page` counter to 1 
%
% \maketitle

{% \usefont{T1}{pnc}{m}{n}
\setlength{\parindent}{0pt}
\thispagestyle{plain}
{\fontsize{18}{20}\selectfont\raggedright 
\maketitle  % title \par  

}

{
   \vskip 13.5pt\relax \normalsize\fontsize{11}{12} 
\textbf{\authorfont Marcio de Lucas} \hskip 15pt \emph{\small Universidade Federal de Pernambuco}   

}

}






\vskip 6.5pt


\noindent \doublespacing \hypertarget{introducao}{%
\section{Introdução}\label{introducao}}

Ao longo das últimas duas decadas, as discussões acerca dos efeitos do
autoritarismo sobre a qualidade e estabilidade das democracias ganharam
maior espaço na academia e na mídia. Contudo, diferente de outros
momentos da história, as principais ameaças atuais para a liberdade são
líderes políticos, eleitos democraticamente, que carregam consigo traços
autoritários nas suas atitudes e comportamento. Em casos mais
dramáticos, como na Venezuela, Hungria e Turquia, a presença de chefes
do executivo pouco simpáticos à democracia foi elemento decisivo para a
ocorrência de reversões autoritárias.

Todavia, apesar da relevância do tema, a literatura ainda conta com
ferramentas limitadas para identificar agentes políticos autoritários.
Em decorrência da dificuldade da coleta de dados primários de membros da
classe política e integrantes de grupos organizados, pesquisadores
utililizam-se de grandes heurísticas como unidade de análise. Alguns dos
exemplos são: partidos políticos (Mudde
\protect\hyperlink{ref-mudde2009populist}{2009},
\protect\hyperlink{ref-mudde2016introduction}{2016}; Loxton
\protect\hyperlink{ref-loxton2014authoritarian}{2014}); movimentos
sociais (Caiani \protect\hyperlink{ref-caiani2017radical}{2017}),
comportamento do eleitorado (Booth and Seligson
\protect\hyperlink{ref-booth1984political}{1984}; Seligson
\protect\hyperlink{ref-seligson2003democracies}{2003}; Seligson and
Tucker \protect\hyperlink{ref-seligson2005feeding}{2005}; Rydgren
\protect\hyperlink{ref-rydgren2007sociology}{2007}) ou grandes
lideranças (Levitsky and Ziblatt
\protect\hyperlink{ref-levitsky2018democracies}{2018}; Norris and
Inglehart \protect\hyperlink{ref-norris2019cultural}{2019}). Como
efeito, o autoritarismo é observado apenas nas suas expressões
organizadas, permanecendo inobservado enquanto atributo individual de
parlamentares e candidatos.

Como forma de propor um critério de classificação para políticos
autoritários, extensível para um grande número de agentes políticos e em
consonância com o conhecimento reunido no campo da psicologia política,
essa dissertação apresenta um modelo de identificação de atitudes
autoritárias baseada na análise automatizada de conteúdo aplicada ao
caso da Câmara dos Deputados. Para isso, coletou-se 420 mil falas de
3700 parlamentares realizadas entre os anos 2000 e 2019. Endoçando os
avanços de trabalhos da Ciência Política que têm utilizado texto como
dado (\emph{text as data}) (Batista and Vieira
\protect\hyperlink{ref-batista2016mensurando}{2016}; Moreira
\protect\hyperlink{ref-moreira2016palavra}{2016}), expera-se contribuir
para a literatura a partir da exploração de fontes de alterativas de
dados para construção de um indicador relevante para a literatura.

A despeito das dificuldades na classificação de políticos autoritários,
uma extensa bibliografia apresenta definições e indicadores extensamente
validados para mensurar propensão ao autoritarismo (Titus and Hollander
\protect\hyperlink{ref-titus1957california}{1957}; Meloen
\protect\hyperlink{ref-meloen1993f}{1993}). De maneira geral, a
personalidade autoritária envolve a crença de que o mundo é um lugar
hostil e periogoso, no qual a segurança coletiva, a estabilidade e a
ordem são indispensáveis a qualquer custo (Altemeyer and Altemeyer
\protect\hyperlink{ref-altemeyer1996authoritarian}{1996}). Partindo
disso, seus representantes tendem a dividir a sociedade em dois grupos
homogêneos: àqueles que buscam preservar o mundo como nós conhecemos e
aderem as autoridades vigentes e aqueles que as desafiam.

Nas investigações empíricas sobre o tema, a literatura reúne evidências
de que a personalidade autoritária está associada à discriminação (Titus
and Hollander \protect\hyperlink{ref-titus1957california}{1957}; Meloen
\protect\hyperlink{ref-meloen1993f}{1993}), preconceito religioso
(Laythe, Finkel, and Kirkpatrick
\protect\hyperlink{ref-laythe2001predicting}{2001}), contra homossexuais
(Hunsberger \protect\hyperlink{ref-hunsberger1996religious}{1996};
Jonathan \protect\hyperlink{ref-jonathan2008influence}{2008}),
preconceito racial (Rowatt and Franklin
\protect\hyperlink{ref-rowatt2004christian}{2004}), à manifestações de
sexismo (Sibley, Wilson, and Duckitt
\protect\hyperlink{ref-sibley2007antecedents}{2007}), xenofobia
(Thomsen, Green, and Sidanius
\protect\hyperlink{ref-thomsen2008we}{2008}) e preconceito em geral
(Asbrock, Sibley, and Duckitt
\protect\hyperlink{ref-asbrock2010right}{2010}). O que o panorama geral
das pesquisas sugere é que, de maneira geral, entidades associadas a
ordem, segurança e autoridade são desproporcionalmente valorizadas por
autoritários e entidades associadas a populações marginais,
desproporcionalmente depreciadas.

Como estratégia metodológica, buscou-se mensurar o valor atribuído, nas
falas dos parlamentares, às duas classes de entidades que possuem grande
saliência no discurso autoritário: os objetos de exaltação e os objetos
de rejeição. Para isso, utilizou-se um dicionário de entidades textuais
como chave para mensurar a valência das sentenças a associada a cada uma
delas. Essa abordagem baseia-se nos trabalhos que utilizaram análise
automatizada de conteúdo para classificar populistas (Rooduijn and
Pauwels \protect\hyperlink{ref-rooduijn2011measuring}{2011}; Oliver and
Rahn \protect\hyperlink{ref-oliver2016rise}{2016}; Aslanidis
\protect\hyperlink{ref-aslanidis2018measuring}{2018}) e identificar
incivilidade (Vargo and Hopp
\protect\hyperlink{ref-vargo2017socioeconomic}{2017}) e depressão
(Neuman et al. \protect\hyperlink{ref-neuman2012proactive}{2012}; Kang,
Yoon, and Kim \protect\hyperlink{ref-kang2016identifying}{2016}).

Os resultados apresentaram correspondência com as expectativas da
literatura. As bacadas da Bala e da Bíblia apresentaram proporcionamente
o maior número de parlamentares autoritários. Como consequência, é
possível perceber, para o caso brasileiro, forte correlação entre
discurso autoritário e o populismo penal. Por fim, outro dado
preocupante, é de que os anos de 2018 e 2019 apresentaram um
significativo aumento na expressão de autoritarismo por meio das falas
de parlamentares.

\newpage

\hypertarget{autoritarismo-na-politica}{%
\section{Autoritarismo na política}\label{autoritarismo-na-politica}}

Num passado não tão distante, autoritarismo e democracia eram
conceituados como antônimos. A publicação do livro
\emph{Capitalism, Socialism and Democracy}, de Schumpeter
(\protect\hyperlink{ref-schumpeter1942capitalism}{1942}), ilustra esse
paradigma. Nele, era enumciada a tese de que a democracia seria
suprimida por governos autocráticos, na medida em que as classes
dominantes -- sejam elas a burguesia ou a vanguarda socialista --
utilizariam-se das restrições a participação popular como forma de
imunizar seu domínio contra a vontade das massas.

O que observou-se após a Terceira Onda de Democratização, todavia, foi
um fenômeno contra-intuitivo: revoluções majoritaristas e
anti-pluralistas trouxeram fins prematuros para algumas das novas
democracias (Lipset
\protect\hyperlink{ref-lipset1993reflections}{1993}). Além disso, um bom
conjunto de achados aponta que: a preferência pela democracia têm se
restringido e a adesão à soluções autoritárias, aumentado, mesmo em
democracias antigas (Foa and Mounk
\protect\hyperlink{ref-foa2016democratic}{2016}; R. S. Foa and Mounk
\protect\hyperlink{ref-foa2017signs}{2017}; R. Foa and Mounk
\protect\hyperlink{ref-foa2017end}{2017}); o apoio à líderes de pulso
firme tem aumentado, consistentemente, ao longo do últimos 20 anos, nas
democracias em desenvolvimento (Voeten
\protect\hyperlink{ref-voeten2016people}{2016}).

O fato de políticos autoritários serem eleitos em regimes democráticos
põem em cheque classificações binárias de sistemas políticos. A ascenção
dessas lideranças a chefia do executivo aumenta as chances de transições
de governo conflituosas, retrocessos na qualidade da democracia, aumento
da percepção de corrupção e erosão de direitos civis (Kyle and Mounk
\protect\hyperlink{ref-mounk2018thepopulist}{2018}). Ou seja, a presença
de agentes autoritários no governo, independente das instituições
políticas de um país, proporciona administrações que se distanciam dos
ideiais democráticos.

Partindo disso, pode-se definir o autoritarismo em três camadas: i) no
nível das configurações do regime, ii) das práticas dos agentes
institucionais e iii) da psicologia dos atores políticos (Glasius
\protect\hyperlink{ref-glasius2018authoritarianism}{2018}). Nesse
sentido, um sistema político é autoritário quando veta eleições, tutela
os direitos políticos dos cidadãos e concentra amplos poderes na elite.
Já no nível intermediário, práticas autoritárias são caracterizadas por
decisões que afetam o \emph{accountability} vertical, na medida em que
limitam a liberdade de expressão efetiva dos cidadãos, controlam o
acesso à informação ou violam a privacidade, por exemplo. Por fim, no
nível individual, políticos podem ser classificados como mais ou menos
autoritários de acordo com a sua adesão aos princípios de que i) a
sociedade deve ser organizada em torno da autoridade e de regras rígidas
e ii) que indivíduos desviantes devem ser punidos.

A diferenciação analítica do fenômeno em níveis, apesar de na prática
eles estarem conectados, é importante para assinalar que o autoritarismo
está presente em todas as formas de organização política, inclusive em
regimes democráticos estabelecidos. Seja nas atitudes do eleitorado ou
comportamento dos parlamentares, sua expressão encontra-se em latência
num grande número de sociedades, suscetível a fatores que estimulam ou
restringem sua emergência.

Do ponto de vista da imagem que agentes políticos autoritários constroem
de si mesmo, suas plataformas de campanha variam significativamente por
contexto cultural. Na Europa Ocidental, partidos de extrema-direita e
direita radical tendem a incorporar nativismo, xenofobia, racismo e
neoliberalismo em seus discursos (Mudde
\protect\hyperlink{ref-mudde2009populist}{2009}). Nos EUA, Nova Zelândia
e Canadá, a política do ressentimento, manifesta pela rejeição ostensiva
ao \emph{establishment}, conjuga a tônica dos \emph{ousiders}
anti-democráticos, e, no caso da Rússia, a xenofobia e o
ultranacionalismo desempenham esse papel (Norris and others
\protect\hyperlink{ref-norris2005radical}{2005}).

No caso Latino-americano, há um grande número de casos de partidos de
extrema-direita que herdaram laços, conexões, recursos e reputação de
regimes autoritários que precederam as transições democráticas que
ocorreram ao longo da década de 70 e 80 (Loxton
\protect\hyperlink{ref-loxton2014authoritarian}{2014}). Em comum,
compartilham a defesa de políticas de segurança \emph{mano dura} como
alavanca de campanha para angariar votos do eleitorado que, a despeito
de antipatizar com as plataformas liberalizantes da economia, enxergava,
nas lideranças linha-dura, qualidades que permitiriam reestabelecer a
ordem sobre o caos gerado pela insegurança.

A Alianza Republicana Nacionalista (ARENA), no Equador, é um caso
especial de sucesso. O partido explorou o vazio de propostas para
resolução dos problemas de segurança e a antipatia gerada pelas
políticas desarmamentistas defendidas pelo campo político progressista
como oportunidade de se posicionar de forma competitiva no cenário
eleitoral (Holland \protect\hyperlink{ref-holland2013right}{2013}). Na
composição do partido, estavam membros dos \emph{esquadrões da morte} --
grupos paramilitares que cooperavam com o governo combatendo guerrilhas
comunistas na guerra civil que se estendeu ao longo da década de 80 na
região -- os quais angariavam credibilidade para execução de políticas
\emph{mano dura} de segurança (Loxton
\protect\hyperlink{ref-loxton2014authoritarian}{2014}).

Apesar do populismo penal ser uma marca de campanha proeminente na
região que concentra maiores estatísticas de violência do mundo, o
autoritarismo é flexível para se adequar em um grande conjunto de
narrativas. No caso da Nicarágua, a família Somoza governou o país ao
longo dos anos de 1934 e 1979, deixando um legado de violações a
direitos civis em nome do combate aos comunistas. Na Venezuela, por
outro lado, sob o governo de Hugo Chavez, estatizações, reforma agrária
e rivalização com neoliberalismo foram medidas implementadas durante o
caminho da reversão autoritária. E em diversas das transições políticas
anti-republicanas que ocorreram na América Latina, setores da sociedade
civil organizada participaram das mobilizações (Valenzuela
\protect\hyperlink{ref-valenzuela2004latin}{2004}).

\hypertarget{a-personalidade-autoritaria}{%
\section{A personalidade
Autoritária}\label{a-personalidade-autoritaria}}

Com tantas cores que preenchem as campanhas políticas de agentes
autoritários, o que define a essência da personalidade autoritária? O
que há de comum entre tantos líderes que frequentemente rivalizaram
entre si?

Uma das primeiras contribuições para responder essa questão foi feita
por Adorno et. al.
(\protect\hyperlink{ref-adorno1950authoritarian}{1950}). Como causa do
fascismo e nazismo que ascenderam como problemáticas sociais após a
Primeira Guerra Mundial, os autores apontaram a propensão ao
autoritarismo como uma psicopatologia desenvolvida no processo de
socialização das crianças, responsável pela manifestação de adesão
crônica à autoridade e rejeição àqueles que estão em posição de
subalternidade.

A escala F, desenvolvida por Adorno como operacionalização da teoria,
mensura a posição de um indivíduo no espectro do autoritarismo. Nas
investigações empíricas sobre o tema, a literatura reúne evidências de
que a escala F está associada à discriminação e adesão ao autoritarismo
(Titus and Hollander \protect\hyperlink{ref-titus1957california}{1957};
Meloen \protect\hyperlink{ref-meloen1993f}{1993}), suspeição e baixa
confiança interpessoal (Deutsch
\protect\hyperlink{ref-deutsch1960trust}{1960}) e atitudes iliberais
(Meloen \protect\hyperlink{ref-meloen1993f}{1993}). Mais recentemente,
Levitsky e Ziblat
(\protect\hyperlink{ref-levitsky2018democracies}{2018}) construíram um
indicador de lideranças de perfil autoritário baseado na medida. Os
autores proveram argumentos e evidências que apontam que chefes do
executivo com elevados escores na escala F representam grande risco para
qualidade e sobrevivência da democracia.

Posteriormente, o indicador de dogmatismo, proposto por Rokeach et. al
(\protect\hyperlink{ref-rokeach1960open}{1960}), interpreta o
autoritarismo como uma forma de etnocentrismo generalizado, no qual
indivíduos de grupos sociais `desviantes' são vistos como uma ameaça.
Fatores como o fundamentalismo religioso, portanto, são identificados
como potenciais causadores do acirramento entre \emph{insiders} e
\emph{outsiders} e, consequentemente, propensão a preconceito religioso
(Laythe, Finkel, and Kirkpatrick
\protect\hyperlink{ref-laythe2001predicting}{2001}), contra homossexuais
(Hunsberger \protect\hyperlink{ref-hunsberger1996religious}{1996};
Jonathan \protect\hyperlink{ref-jonathan2008influence}{2008}) e racial
(Rowatt and Franklin \protect\hyperlink{ref-rowatt2004christian}{2004}).

Porém, foram com os trabalhos de Altemeyer
(\protect\hyperlink{ref-altemeyer1981right}{1981}) que os efeitos
cognitivos da violência sobre o autoritarismo ficaram mais claros. Em
sua principal contribuição para a literatura, definiu o \emph{Right-wing
Authoritarianism} (RWA) como uma medida da percepção do mundo como um
lugar hostil e perigoso, no qual a segurança coletiva, a estabilidade e
a ordem são indispensáveis a qualquer custo.

Nessa perspectiva, o autorarismo é subproduto de um estado mental de
superestresse, ocasiodo por uma profunda sensação de medo e impotência.
Como resposta ao contexto em que se encontram, indivíduos tendem a
adotar uma postura de suspeição radical. Essa postura combina uma
submissão dogmática às autoridades, às regras e às convenções sociais
que sustentam a realidade social como conhecida com uma resposta
agressiva com o desconhecido.

Uma vez que a socialização primária oferece ambiente seguro para o
primeiro contato com preferências e atitudes, autoritários tendem a
resgatar seu comportamento associativo mais primário, emulando relações
patriarcalistas. Ou seja, de forma análoga à regra de Hamilton, em
contextos de elevada insegurança, indivíduos tendem a priorizar seus
valores e relações com pares próximos, como forma de garantir seu
investimento emocional nas interações em contexto de risco, a despeito
disso implicar em manter uma teia de interações mais restrita e
hierarquizada (Ohtsuki et al.
\protect\hyperlink{ref-ohtsuki2006cooperation}{2006}).

Tomando o RWA como preditor, trabalhos empíricos sobre tema apontam que
indivíduos que apresentam maiores valores de RWA estão mais propensos à
manifestações de sexismo (Sibley, Wilson, and Duckitt
\protect\hyperlink{ref-sibley2007antecedents}{2007}), xenofobia
(Thomsen, Green, and Sidanius
\protect\hyperlink{ref-thomsen2008we}{2008}) e preconceito em geral
(Asbrock, Sibley, and Duckitt
\protect\hyperlink{ref-asbrock2010right}{2010}). Além disso, outras
evidências sugerem que RWA está associado à defesa do uso de violência
de grupos dominantes sobre dominados (Henry et al.
\protect\hyperlink{ref-henry2005social}{2005}) e maior aceitação do uso
de agressão em guerras, na punição de criminosos e na educação das
crianças (Benjamin
\protect\hyperlink{ref-benjamin2006relationship}{2006}).

Numa anedota expressa por

\newpage

\hypertarget{referencias}{%
\section*{Referências}\label{referencias}}
\addcontentsline{toc}{section}{Referências}

\hypertarget{refs}{}
\leavevmode\hypertarget{ref-adorno1950authoritarian}{}%
Adorno, Theodor W, Else Frenkel-Brunswik, Daniel J Levinson, R Nevitt
Sanford, Betty Ruth Aron, Maria Hertz Levinson, and William R.. Morrow.
1950. \emph{The Authoritarian Personality}. Harper New York.

\leavevmode\hypertarget{ref-altemeyer1981right}{}%
Altemeyer, Bob. 1981. \emph{Right-Wing Authoritarianism}. University of
Manitoba press.

\leavevmode\hypertarget{ref-altemeyer1996authoritarian}{}%
Altemeyer, Robert A, and Bob Altemeyer. 1996. \emph{The Authoritarian
Specter}. Harvard University Press.

\leavevmode\hypertarget{ref-asbrock2010right}{}%
Asbrock, Frank, Chris G Sibley, and John Duckitt. 2010. ``Right-Wing
Authoritarianism and Social Dominance Orientation and the Dimensions of
Generalized Prejudice: A Longitudinal Test.'' \emph{European Journal of
Personality: Published for the European Association of Personality
Psychology} 24 (4). Wiley Online Library: 324--40.

\leavevmode\hypertarget{ref-aslanidis2018measuring}{}%
Aslanidis, Paris. 2018. ``Measuring Populist Discourse with Semantic
Text Analysis: An Application on Grassroots Populist Mobilization.''
\emph{Quality \& Quantity} 52 (3). Springer: 1241--63.

\leavevmode\hypertarget{ref-batista2016mensurando}{}%
Batista, Mariana, and Bhreno Vieira. 2016. ``Mensurando Saliência: Uma
Medida Com Base Em ênfase Na Agenda Legislativa Do Brasil
(1995--2014).'' \emph{Primeira Versão. Manuscrito Apresentado Para X
ABCP}.

\leavevmode\hypertarget{ref-benjamin2006relationship}{}%
Benjamin, Arlin James. 2006. ``The Relationship Between Right-Wing
Authoritarianism and Attitudes Toward Violence: Further Validation of
the Attitudes Toward Violence Scale.'' \emph{Social Behavior and
Personality: An International Journal} 34 (8). Scientific Journal
Publishers: 923--26.

\leavevmode\hypertarget{ref-booth1984political}{}%
Booth, John A, and Mitchell A Seligson. 1984. ``The Political Culture of
Authoritarianism in Mexico: A Reexamination.'' \emph{Latin American
Research Review} 19 (1). JSTOR: 106--24.

\leavevmode\hypertarget{ref-caiani2017radical}{}%
Caiani, Manuela. 2017. ``Radical Right-Wing Movements: Who, When, How,
and Why?'' \emph{Sociopedia. Isa}, 1--15.

\leavevmode\hypertarget{ref-deutsch1960trust}{}%
Deutsch, Morton. 1960. ``Trust, Trustworthiness, and the F Scale.''
\emph{The Journal of Abnormal and Social Psychology} 61 (1). American
Psychological Association: 138.

\leavevmode\hypertarget{ref-foa2017end}{}%
Foa, R, and Yascha Mounk. 2017. ``The End of the Consolidation
Paradigm.'' \emph{Journal of Democracy Web Exchange}.

\leavevmode\hypertarget{ref-foa2016democratic}{}%
Foa, Roberto Stefan, and Yascha Mounk. 2016. ``The Democratic
Disconnect.'' \emph{Journal of Democracy} 27 (3). Johns Hopkins
University Press: 5--17.

\leavevmode\hypertarget{ref-foa2017signs}{}%
---------. 2017. ``The Signs of Deconsolidation.'' \emph{Journal of
Democracy} 28 (1). Johns Hopkins University Press: 5--15.

\leavevmode\hypertarget{ref-glasius2018authoritarianism}{}%
Glasius, Marlies. 2018. ``What Authoritarianism Is\ldots{} and Is Not: A
Practice Perspective.'' \emph{International Affairs} 94 (3). Oxford
University Press: 515--33.

\leavevmode\hypertarget{ref-henry2005social}{}%
Henry, Patrick J, Jim Sidanius, Shana Levin, and Felicia Pratto. 2005.
``Social Dominance Orientation, Authoritarianism, and Support for
Intergroup Violence Between the Middle East and America.''
\emph{Political Psychology} 26 (4). Wiley Online Library: 569--84.

\leavevmode\hypertarget{ref-holland2013right}{}%
Holland, Alisha C. 2013. ``RIGHT on Crime? Conservative Party Politics
and" Mano Dura" Policies in El Salvador.'' \emph{Latin American Research
Review}. JSTOR, 44--67.

\leavevmode\hypertarget{ref-hunsberger1996religious}{}%
Hunsberger, Bruce. 1996. ``Religious Fundamentalism, Right-Wing
Authoritarianism, and Hostility Toward Homosexuals in Non-Christian
Religious Groups.'' \emph{The International Journal for the Psychology
of Religion} 6 (1). Taylor \& Francis: 39--49.

\leavevmode\hypertarget{ref-jonathan2008influence}{}%
Jonathan, Eunike. 2008. ``The Influence of Religious Fundamentalism,
Right-Wing Authoritarianism, and Christian Orthodoxy on Explicit and
Implicit Measures of Attitudes Toward Homosexuals.'' \emph{The
International Journal for the Psychology of Religion} 18 (4). Taylor \&
Francis: 316--29.

\leavevmode\hypertarget{ref-kang2016identifying}{}%
Kang, Keumhee, Chanhee Yoon, and Eun Yi Kim. 2016. ``Identifying
Depressive Users in Twitter Using Multimodal Analysis.'' In \emph{2016
International Conference on Big Data and Smart Computing (Bigcomp)},
231--38. IEEE.

\leavevmode\hypertarget{ref-mounk2018thepopulist}{}%
Kyle, Jordan, and Yascha Mounk. 2018. ``The Populist Harm to Democracy:
An Empirical Assessment.'' \emph{Tony Blair Institut for Global Change}.

\leavevmode\hypertarget{ref-laythe2001predicting}{}%
Laythe, Brian, Deborah Finkel, and Lee A Kirkpatrick. 2001. ``Predicting
Prejudice from Religious Fundamentalism and Right-Wing Authoritarianism:
A Multiple-Regression Approach.'' \emph{Journal for the Scientific Study
of Religion} 40 (1). Wiley Online Library: 1--10.

\leavevmode\hypertarget{ref-levitsky2018democracies}{}%
Levitsky, Steven, and Daniel Ziblatt. 2018. \emph{How Democracies Die}.
Crown.

\leavevmode\hypertarget{ref-lipset1993reflections}{}%
Lipset, Seymour Martin. 1993. ``Reflections on Capitalism, Socialism \&
Democracy.'' \emph{Journal of Democracy} 4 (2). Johns Hopkins University
Press: 43--55.

\leavevmode\hypertarget{ref-loxton2014authoritarian}{}%
Loxton, James Ivor. 2014. ``Authoritarian Inheritance and Conservative
Party-Building in Latin America.'' PhD thesis.

\leavevmode\hypertarget{ref-meloen1993f}{}%
Meloen, Jos D. 1993. ``The F Scale as a Predictor of Fascism: An
Overview of 40 Years of Authoritarianism Research.'' In \emph{Strength
and Weakness}, 47--69. Springer.

\leavevmode\hypertarget{ref-moreira2016palavra}{}%
Moreira, Davi Cordeiro. 2016. ``Com a Palavra Os Nobres Deputados:
Frequência E ênfase Temática Dos Discursos Dos Parlamentares
Brasileiros.'' PhD thesis, Universidade de São Paulo.

\leavevmode\hypertarget{ref-mudde2009populist}{}%
Mudde, Cas. 2009. \emph{Populist Radical Right Parties in Europe}.
Cambridge University Press Cambridge.

\leavevmode\hypertarget{ref-mudde2016introduction}{}%
---------. 2016. ``Introduction to the Populist Radical Right.'' In
\emph{The Populist Radical Right}, 22--35. Routledge.

\leavevmode\hypertarget{ref-neuman2012proactive}{}%
Neuman, Yair, Yohai Cohen, Dan Assaf, and Gabbi Kedma. 2012. ``Proactive
Screening for Depression Through Metaphorical and Automatic Text
Analysis.'' \emph{Artificial Intelligence in Medicine} 56 (1). Elsevier:
19--25.

\leavevmode\hypertarget{ref-norris2019cultural}{}%
Norris, Pippa, and Ronald Inglehart. 2019. \emph{Cultural Backlash:
Trump, Brexit, and Authoritarian Populism}. Cambridge University Press.

\leavevmode\hypertarget{ref-norris2005radical}{}%
Norris, Pippa, and others. 2005. \emph{Radical Right: Voters and Parties
in the Electoral Market}. Cambridge University Press.

\leavevmode\hypertarget{ref-ohtsuki2006cooperation}{}%
Ohtsuki, Hisashi, Christoph Hauert, Erez Lieberman, and Martin A Nowak.
2006. ``A Simple Rule for the Evolution of Cooperation on Graphs and
Social Networks.'' \emph{Nature} 441 (7092). Nature Publishing Group:
502--5.

\leavevmode\hypertarget{ref-oliver2016rise}{}%
Oliver, J Eric, and Wendy M Rahn. 2016. ``Rise of the Trumpenvolk:
Populism in the 2016 Election.'' \emph{The ANNALS of the American
Academy of Political and Social Science} 667 (1). SAGE Publications Sage
CA: Los Angeles, CA: 189--206.

\leavevmode\hypertarget{ref-rokeach1960open}{}%
Rokeach, Milton, and others. 1960. ``The Open and Closed Mind.'' Basic
books New York.

\leavevmode\hypertarget{ref-rooduijn2011measuring}{}%
Rooduijn, Matthijs, and Teun Pauwels. 2011. ``Measuring Populism:
Comparing Two Methods of Content Analysis.'' \emph{West European
Politics} 34 (6). Taylor \& Francis: 1272--83.

\leavevmode\hypertarget{ref-rowatt2004christian}{}%
Rowatt, Wade C, and Lewis M Franklin. 2004. ``Christian Orthodoxy,
Religious Fundamentalism, and Right-Wing Authoritarianism as Predictors
of Implicit Racial Prejudice.'' \emph{The International Journal for the
Psychology of Religion} 14 (2). Taylor \& Francis: 125--38.

\leavevmode\hypertarget{ref-rydgren2007sociology}{}%
Rydgren, Jens. 2007. ``The Sociology of the Radical Right.'' \emph{Annu.
Rev. Sociol.} 33. Annual Reviews: 241--62.

\leavevmode\hypertarget{ref-schumpeter1942capitalism}{}%
Schumpeter, Joseph Alois, and others. 1942. ``Capitalism, Socialism, and
Democracy.'' Harper.

\leavevmode\hypertarget{ref-seligson2003democracies}{}%
Seligson, Amber Lara. 2003. ``When Democracies Elect Dictators:
Motivations for and Impact of the Election of Former Authoritarians in
Argentina and Bolivia.''

\leavevmode\hypertarget{ref-seligson2005feeding}{}%
Seligson, Amber L, and Joshua A Tucker. 2005. ``Feeding the Hand That
Bit You: Voting for Ex-Authoritarian Rulers in Russia and Bolivia.''
\emph{Demokratizatsiya} 13 (1). Citeseer.

\leavevmode\hypertarget{ref-sibley2007antecedents}{}%
Sibley, Chris G, Marc S Wilson, and John Duckitt. 2007. ``Antecedents of
Men's Hostile and Benevolent Sexism: The Dual Roles of Social Dominance
Orientation and Right-Wing Authoritarianism.'' \emph{Personality and
Social Psychology Bulletin} 33 (2). Sage Publications Sage CA: Thousand
Oaks, CA: 160--72.

\leavevmode\hypertarget{ref-thomsen2008we}{}%
Thomsen, Lotte, Eva GT Green, and Jim Sidanius. 2008. ``We Will Hunt
Them down: How Social Dominance Orientation and Right-Wing
Authoritarianism Fuel Ethnic Persecution of Immigrants in Fundamentally
Different Ways.'' \emph{Journal of Experimental Social Psychology} 44
(6). Elsevier: 1455--64.

\leavevmode\hypertarget{ref-titus1957california}{}%
Titus, H Edwin, and Edwin P Hollander. 1957. ``The California F Scale in
Psychological Research: 1950-1955.'' \emph{Psychological Bulletin} 54
(1). American Psychological Association: 47.

\leavevmode\hypertarget{ref-valenzuela2004latin}{}%
Valenzuela, Arturo. 2004. ``Latin American Presidencies Interrupted.''
\emph{Journal of Democracy} 15 (4). Johns Hopkins University Press:
5--19.

\leavevmode\hypertarget{ref-vargo2017socioeconomic}{}%
Vargo, Chris J, and Toby Hopp. 2017. ``Socioeconomic Status, Social
Capital, and Partisan Polarity as Predictors of Political Incivility on
Twitter: A Congressional District-Level Analysis.'' \emph{Social Science
Computer Review} 35 (1). SAGE Publications Sage CA: Los Angeles, CA:
10--32.

\leavevmode\hypertarget{ref-voeten2016people}{}%
Voeten, Erik. 2016. ``Are People Really Turning Away from Democracy?''
\emph{Journal of Democracy Web Exchange}.
\newpage
\singlespacing 
\end{document}
